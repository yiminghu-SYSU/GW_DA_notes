%# -*- coding: utf-8-unix -*-
\begin{overview}
\thispagestyle{empty}

这份讲义是我为中山大学物理与天文学院研究生课程“引力波数据处理”课程所准备的。
由于本人能力有限,准备时间仓促,一定包含了大量错误,我会力争在收到反馈之后进行更正。
讲义的电子版可以在\url{https://github.com/yiminghu-SYSU/GW_DA_notes}获得。

这本讲义的写作对象是对引力波数据处理感兴趣的高年级本科生或研究生。
本书默认读者已经初步掌握了狭义相对论的基本概念,并有一定概率论基础。
在讲义中,我会尽可能追求内容的完整性,以便尚未完成广义相对论等课程学习的同学也可以在脑中构建起足够的物理图景。

但限于篇幅和授课计划,本书肯定无法替代广义相对论等基础课程。
因此,文中肯定会在数学逻辑的严谨性和课程主题的完备性之间做出倾向于后者的取舍,也敬请诸位谅解。

  \rightline{\kaishu{胡一鸣}}
  \rightline{\kaishu{2019年8月14日,于珠海唐家}}
\end{overview}
