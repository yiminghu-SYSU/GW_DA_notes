%# -*- coding: utf-8-unix -*-
%%==================================================
\chapter{相对论基础}

本讲义的授课主题,是\gw\DA,共分为两部分,\emph{\gw}与\emph{\DA}。
如果脱离了\gw 的物理图景,而直接空谈\DA,未免空中楼阁。
而在另一方面,引力波又是Einstein广义相对论的直接理论预言,因此,引力波的理论描述,无法跳脱广义相对论的框架。

\begin{figure}[htp]
\centering
\includegraphics[width=0.7\textwidth]{ModifiedGravity.png}
\bicaption{修改引力理论}
  {Theories of modified gravity. Credit: http://www.cgc-yzu.cn/Upload/research/MG-20240317524.png}
\label{fig:ModGrav}
\end{figure}

从Einstein至今,引力理论已经有了长足的发展,如图\ref{fig:ModGrav}所示,仅基于\GR 基础上发展起来的修改引力理论就已不计其数。
由于和量子力学原理的深刻矛盾,有理由认为Einstein决定论性的的\GR 在某个地方一定背离了引力的物理本质。
然而,时至今日,Einstein昔日基于\GR 所作出的诸多预言,一一被实验所验证;所有可靠的实验检验下,\GR 均可以给出解释——而它通常是最简洁的那个理论。
因此,即使将来的实验证明了\GR 与引力的物理本质之间的偏离,对\GR 的理解依然有着重要的意义。

\section{相对性原理(Principle of relativity)}

\subsection{Galilean相对论}
虽然在20世纪,相对论一次专指Einstein的理论,但是相对性原理(Principle of relativity)的思想在Newtonian力学中就有体现:两个服从Newtonian力学的、相对均匀运动的惯性参考系,无法通过在内部展开的力学实验进行区分。
这一思想一般认为是Galileo在《关于Ptolemaic和 Copernican两大世界体系的对话》中首先提出的\cite{Fang2012blog}:
\begin{myprop}{}{}
把你和一些朋友关在一条大船的甲板下的主舱里,让你们带着几支苍蝇、蝴蝶和其他小飞虫,舱内放一支大碗,其中有几条鱼,然后,挂上一个水瓶,让水一滴一滴地滴到下面的一个宽口罐里。船停着不动时,你留神观察,小虫都以等速向舱内各方向飞行,鱼向各方向随便游动,水滴滴进下面的罐中。你把任何东西扔给你的朋友时,只要距离相等,向这一方向也不比向另一方向更多用力。你的双脚齐跳,无论向哪个方向跳过的距离都相等。当你仔细观察这些事情之后,再使船以任何速度前进,只要运动是均匀的,也不忽左忽右地摆动,你将发现,所有上述现象都没有丝毫变化,你无法从任何一个现象来确定,船是在运动还是在停着不动。即使船运动得相当快,在跳跃时,你也将和以前一样,你跳向船尾也不会比跳向船头更省力。
\end{myprop}
在Galilean相对性原理表明的这个表述中,日常生活中涉及到的物理学性质,在地球坐标系下(船停着不动)和船的坐标系下(船在运动)没有任何区别。

用公式来表述的话,则可以设立两个坐标系,亦即“静止的”地球坐标系S(t,x,y,z)和“运动的”船坐标系S'(t',x',y',z')。
不妨令$t=t'=0$时,两个坐标系重合,且船以速度v沿x方向移动,则有
\begin{equation}
\begin{array}{r@{}l}
t' &{}= t \\
x' &{}= x-vt\\
y' &{}= y \\
z' &{}= z 
\end{array}\label{eq:galileo}
\end{equation}
这种变换通常被称为Galilean变换。

实际上,这种朴素的相对论性原理是非常直观的,在《尚书纬·考灵曜(y{\` a}o)》中,就有文字表达了相当类似的想法:
\begin{myprop}{}{}
地恒动不止,而人不知,如坐闭牖(y{\v o}u)舟中,舟行而人不觉也。
\end{myprop}

\subsection{Maxwell方程组}
通过对电与磁的性质的研究,Maxwell总结了一组著名的方程,用以描述电磁场的一般性质。
在真空中,可以记为
\begin{equation}
\begin{array}{r@{}l}
\nabla \cdot \mathbf {E} &{}= {\frac {\rho }{\varepsilon _{0}}}\\
\nabla \cdot \mathbf {B} &{}= 0\\
\nabla \times \mathbf {E}&{}=-{\frac {\partial \mathbf {B} }{\partial t}}\\
\nabla \times \mathbf {B}&{}=\mu _{0}\left(\mathbf {J} +\varepsilon _{0}{\frac {\partial \mathbf {E} }{\partial t}}\right)
\end{array}\label{eq:maxwell}
\end{equation}
其中$\varepsilon_0$为真空电容率,$\mu_0$为真空磁导率。

Maxwell注意到,通过Maxwell方程组,可以推导出,
\begin{equation}\label{eq:EMwave}
\nabla ^2 \mathbf {B} - \varepsilon_{0} \mu_0 \frac {\partial^2 \mathbf {B} }{\partial^2 t}= 0
\end{equation}
不难看出,电磁场的变化以波动形式传播,其速度$c$取决于:
\begin{equation}\label{eq:SpeedOfLight}
c^2 = \frac{1}{\varepsilon_0 \mu_0}
\end{equation} 
从数值上,$c$的取值与当时已经从实验上测得的光速极为接近,这使得他大胆假设:光就是一种电磁波。

然而,Maxwell方程组与Galilean变换是不自洽的。
考虑在运动的船S'上进行电磁学测量,根据Galilean变换\ref{eq:galileo},电磁场的传播方程\ref{eq:EMwave}变为

\begin{equation}
c^2\nabla^2 \textbf B = \frac{\partial^2 \textbf B}{\partial t^2} + (\textbf v \cdot \nabla)^2 \textbf B - 2 \textbf v \cdot \nabla \left(\frac{\partial \textbf B}{\partial t}\right)
\end{equation}

一个平庸的结论是,通过Galilean变换,船上的物理学家将测得光速变为$c\pm v$。
然而更深刻的问题是,这一结论意味着,如果Galilean相对论是正确的,那么Maxwell方程组只能对某个特定惯性参考系成立,而物理学家可以根据电磁场的测量来确定实验室位于“船”上还是相对静止。
Newtonian力学必须借助绝对绝对空间的概念,在坚持Galilean相对论的前提下,似乎可以得出,满足\ref{eq:EMwave}的参考系就是Newtonian力学概念中的绝对空间。

其时,人们认为电磁波传播需要介质,而这种依附于绝对空间而具有独特性质的参考系,具象化为“以太(aether)”\cite{EleDyn1997}。

\subsection{\SR}
Lorentz 和Poincar{\'e}第一次意识到,如果S坐标系和S'坐标系之间的转换关系采用如
\begin{equation}
\begin{array}{r@{}l}
t'&{}=\gamma \left(t-{\frac {vx}{c^{2}}}\right)\\
x'&{}=\gamma \left(x-vt\right)\\
y'&{}=y\\
z'&{}=z\\
  \gamma &{}= \frac{1}{{\sqrt {1-v^{2}/c^{2}}}}
\end{array}\label{eq:lorentz}
\end{equation}
的形式的话,那么Maxwell方程组(公式\ref{eq:maxwell})在所有的惯性系中都能成立。
形如公式\ref{eq:lorentz}的变换称为Lorentz变换(Lorentz transformation),我们可以说,Maxwell方程组在不同的惯性系中,通过Lorentz变换维持了不变性。

从某种意义上说,Lorentz变换就是\SR 的精髓。
但物理学界今天达成共识,认为是Einstein而非Lorentz或Poincar{\'e}发明了狭义相对论,这是因为Einstein第一次严肃地认为Lorentz变换体现了时间与空间的本质,而非简单的数学玩具。
由此,时间与空间并非完全独立,而是交织在一起,甚至可以互换。
要标记一个事件,必须同时标记其在某个惯性系下的空间坐标(x,y,z)和时间坐标t。
对于两个事件,在S坐标系下看来可能是同时发生的($\Delta t=0$),但在S'坐标系看来却可能发生于不同时间($\Delta t \neq 0$)。
在不同的惯性系下,通过Lorentz变化,两个不同事件之间,保持不变的,是事件间时间间隔和空间间隔的某种组合,称为时空间隔离:
\begin{equation}\label{eq:STinterval}
  \Delta s^2 = - (c\Delta t)^2+ \Delta x ^2 + \Delta y^2+ \Delta z^2
\end{equation}
通常其微分形式使用起来更为方便:
\begin{equation}\label{eq:STinterval_diff}
  {\rm d}s^2 = - (c{\rm d}t)^2+ {\rm d}x ^2 + {\rm d} y^2+ {\rm d} z^2
\end{equation}
在Lorentz变换下,${\rm d}s^2$保持不变。

\subsection{张量初步}
在相对论框架下,时间坐标t和空间坐标(x,y,z)联合起来形成一个统一的时空坐标
$x^{\alpha} = (ct,x,y,z)$。
如此处的$\alpha$一般出现在坐标上标上的希腊字母,会遍历{0,1,2,3}。
$x^0= ct$的选取是使得所有的坐标都有着相同的空间量纲,而$x^1 =x, x^2 = y, x^3 = z$则代表空间坐标。
这样,我们可以将公式\ref{eq:STinterval_diff}重新表达为
\begin{equation}\label{eq:STinterval_diff_new}
  {\rm d}s^2 = \eta_ {\alpha \beta}{\rm d}x^\alpha {\rm d} x^\beta
\end{equation}
这里,$\eta_ {\alpha \beta}$是一个对角矩阵,
\begin{equation}\label{eq:Minkowski}
  \eta_{\alpha\beta} ={\begin{pmatrix}
                       -1 & 0 & 0 & 0\\
                        0 & 1 & 0 & 0\\ 
                        0 & 0 & 1 & 0\\
                        0 & 0 & 0 & 1\end{pmatrix}}
\end{equation}
注意到$\eta_{00} = -1$, 而$\eta_{11} = \eta_{22} = \eta_{33} = 1$,并且,根据{\heiti{Einstein 求和约定(Einstein summation convention)}},如公式\ref{eq:STinterval_diff_new}一般,当某个希腊字母同时出现在上下标的时候,则意味着要对该字母求和。

通过公式\ref{eq:STinterval_diff_new},我们可以把随坐标变换而改变的${\rm }x^{\alpha} $ 转化成不随坐标变换而改变的时空间隔${\rm d} s^2$
