%# -*- coding: utf-8-unix -*-
%%==================================================

\chapter{引力波信号探测}
\label{chap6}

\section{概率初步}
\subsection{随机变量}
% Jaranowski book Ch 3.1
% Gregory 1-2
% Sivia 1 
\subsection{频率学派v.s.贝叶斯学派}
% Creighton and Anderson 7.2.1
\subsubsection{最大熵原理}
% Gregory 8.7
\subsection{典型分布}
% Jaranowski book Ch 3.1
\subsubsection{Binomial}
\subsubsection{Poisson}
\subsubsection{Gaussian}
% Creighton and Anderson 7.1.2
\subsection{随机过程}
% Jaranowski book Ch 3.2
% Creighton and Anderson 7.1

\section{时序列分析}
\subsection{样本平均和相关函数}
% Jaranowski book Ch 4.1
\subsection{功率谱密度}
% Jaranowski book Ch 4.2 
% Creighton and Anderson 7.1.1
% Finn 92

\section{信号探测的统计学原理}
% Jaranowski LRR Ch 3
\subsection{假设检验}
% Jaranowski LRR Ch 3.1
% Finn 92
\subsubsection{频率学派方法}
\subsubsection{贝叶斯学派方法}
\subsubsection{Neyman-Pearson方法}
\subsubsection{似然函数比检验}

\section{连续引力波探测}
\subsection{F-统计}
% Jaranowski LRR Ch 4.1
\subsection{误警率和探测概率}
% Jaranowski LRR Ch 4.3
\subsection{模板数}
% Jaranowski LRR Ch 4.4
% Allen 2019 
\subsection{次优滤波}
% Jaranowski LRR Ch 4.5
\subsection{次优滤波}

\section{啁啾信号探测}
\subsection{最佳探测统计}
% Finn 92
% Allen 05 (\rho_new) https://journals.aps.org/prd/pdf/10.1103/PhysRevD.71.062001
% Jaranowski LRR Ch 3.2
\subsection{匹配滤波}
% Creighton and Anderson 7.2.2
\subsubsection{未知匹配参数}
% Creighton and Anderson 7.2.3
\subsubsection{匹配滤波的统计学性质}
% Creighton and Anderson 7.2.4
\subsubsection{时间未知的匹配滤波}
% Creighton and Anderson 7.2.5
\subsubsection{匹配滤波模板库}
% Creighton and Anderson 7.2.6
\subsection{显著度分析}
% Capano et al. 2017 PRD
%http://pycbc.org/pycbc/latest/html/credit.html
