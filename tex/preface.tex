
\begin{pre}
	\thispagestyle{empty}
          {\kaishu{
          
          Joseph Weber是引力波探测的先驱。他建造了两个相隔上千公里的探测器,这样就可以有效地隔绝环境噪声产生的误警(实际上,LIGO也是遵从这样的宗旨选址的),因为引力波的传播速度是光速,所以两个探测器应该近乎同时探测到信号;而噪声的到来对不同的探测器则是相互独立的。Weber记录的数据中,就有几个信号,不同的探测器记录的时间差非常短。
          1969年,Weber在{\emph Physical Review Letters}上发表论文“Evidence for discovery of gravitational radiation”,宣称发现了引力波信号。一时间,他名声鹊起,成为了聚光灯前的宠儿,媒体将其捧为继Einstein之后最伟大的物理学家。

          2017年,三位物理学家由于引力波的发现而获得诺贝尔奖。三个人里,没有Weber。当然,斯人已去,Weber在世纪之交的2000年9月30日就已作古。然而,Weber与诺奖失之交臂,并不是因为他活的不够长——这一点日渐成为获得诺奖的必要非充分条件——而是因为后续的研究无法重复Weber的结论。先是理论计算结果与Weber宣称的引力波信号强度、频次相差悬殊;接着,其他团队建造的更灵敏的探测器宣布无法探测到引力波事件;最致命的是,Weber在相隔三个时区的两个信号中寻找同时信号的时候,竟然忘了考虑时差。

          Weber是真诚的,直到他去世时,他依然坚信他成功地探测到了引力波。这里并没有学术不端,有的,只是对数据处理的极度忽视。当然,也许,Weber数据处理地好一些,就不会引发这么大的轰动,也就不会吸引这么多聪明人投身到引力波探测这个领域,也许2015年9月14日穿越地球而过的那一串小小涟漪,就会如它之前的所有双黑洞并合的信号般,悄悄地来,又悄悄地去,不留下意思痕迹。

          2016年,引力波探测的大门,已然被叩开,黑洞与中子星疯狂的舞蹈终于觅来了知音。在2019年的这个夏天,当我远眺未来,充满的是憧憬和期待,当天琴卫星上天,当宇宙的长波电台被天琴接收到信号时,又会上演什么样的一出好戏呢?科学的魅力往往就在于它的不可捉摸和不可预测。不管是什么样的发现,背后一定会有着有力的数据处理方法作为支撑。还好,卫星上的时钟同步用的是原子钟,应该不至于忘记考虑时差这回事。

          历史不容假设,当我们回望过去的时候,我们也许会将Weber铭记为一位勇敢的先驱,一位卓绝的工程师,甚或是一位天才的实验物理学家;但同时,不容否认,他在引力波数据处理这门课上,表现地糟糕透了。
          我相信,当未来的人们回望这段当前,回顾中国科学的崛起时,引力波探测、天琴计划,都会是绕不过去的、在历史的长河中熠熠发光的名字。我们有着国内顶尖的激光团队,国内顶尖的惯性基准团队,国内顶尖的卫星系统团队。我希望,我们也将会有国内顶尖的引力波数据处理团队。

          这段历史画卷,现在就将由你谱写。
          愿这一本讲义,能成为你的画笔,你的颜料。

          愿原力与你同在。
          }}

%\vspace*{5\baselineskip}
%\centerline{\includegraphics[scale=0.6]{example/gzh.jpg}}
%\centerline{\fontsize{26pt}{26pt} 微信公众号}
\end{pre}
