%# -*- coding: utf-8-unix -*-
%%==================================================

\chapter{引力波探测手段}
\label{chap4}

\section{棒状探测器}
\subsection{原理}
% Sathya and Schutz 4.1
% Saulson 13.2-13.4
% Schutz p215
% Maggorie 8.4
% Creighton and Anderson 6.4

\subsection{噪声}
% Saulson 13.4,13.5,13.6
% Schutz p215

\subsection{灵敏度}
% Saulson 13.7,13.8,13.9, 15.3.1, 15.3.2, 15.3.3


\section{地面激光干涉探测器}
\subsection{原理}
% Will 11.5
% Sathya and Schutz 4.2
% Schutz p220
% Simon ppt
% Jaranowski 5.1-5.2
% Saulson 6.3-6.5, 6.9
% Maggorie 9.1-9.2
% Creighton and Anderson 6.1.1-6.1.3 (material 6.1.4-6.1.6)

\subsection{波源和噪声}
% Sathya and Schutz 4.3
% Schutz p220
% Simon ppt
% Creighton and Anderson 6.1.8-6.1.11
% Saulson 7 8 9
% Maggorie 9.1-9.2

\subsection{灵敏度}
% Sathya and Schutz 4.5-4.6
% 1604.00439
% wiki of LIGO detections/ separate for CW and stochastic

\section{空间引力波探测}
\subsection{原理}
% Mei internal document
% Creighton and Anderson 6.2
% Simon ppt
% Simon phd thesis
% Moore et al. CQG
\subsection{波源和噪声}
% Simon ppt
% Simon phd thesis
\subsection{现状}
% GW_Study_Rev3_Aug2012-Final
% BBO paper
% DECIGO paper
% TaiJI paper

\section{脉冲星计时阵}
\subsection{原理}
% Hobbs review
% Creighton and Anderson 6.3
% Maggorie 6.2
\subsection{源、噪声和现状}
% Hobbs review

\section{宇宙微波背景辐射}
\subsection{原理}
% Kamionkowski review
\subsection{源、噪声和现状}
% Kamionkowski review

\section{其他探测方案}
% LSD from Northwestern
% chongqing uni?
% Ando's proposal?
% atom interferometry
