%# -*- coding: utf-8-unix -*-
%%==================================================

\chapter{其他方法及复杂情形}
\label{chap8}

\section{随机引力波背景探测}
% Romano and Cornish LRR 4
\subsection{单探测器信噪比}
% Maggorie1 7.8.2
\subsection{相关与似然函数}
% Romano and Cornish LRR 4.2
\subsection{多数据情形}
% Romano and Cornish LRR 4.3
% Maggorie1 7.8.3
\subsection{最大似然探测统计}
% Romano and Cornish LRR 4.4
\subsection{贝叶斯相关分析}
% Romano and Cornish LRR 4.5
\subsubsection{贝叶斯相关分析与频率派互相关方法比较}
% Romano and Cornish LRR 4.6
\subsection{其他方法}
% Romano and Cornish LRR 4.7

\section{无法建模信号的探测统计量}
% Creighton and Anderson 7.4
% Barack I 7.1.3
\subsection{功率超出法}
% Creighton and Anderson 7.4.1

\section{机器学习}
% George and Heurta doi.org/10.1016/j.physletb.2017.12.053

\section{非稳态、非Gauss、非线性噪声下的探测}
% Creighton and Anderson 7.5
% Jaranowski LRR Ch 6

\section{数据包含间隔的情形}
% Baghi et al. 2019 (arXiv:1907.04747)

\section{效率提高}
\subsection{Reduced-ordermodels}
% Barack II 7.3.1
\subsection{Kludge}
% Barack II 7.3.2
