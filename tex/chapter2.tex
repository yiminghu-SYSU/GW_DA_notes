%# -*- coding: utf-8-unix -*-
%%==================================================

\chapter{引力波方程}
\label{chap2}

\section{\GR 的Newtonian极限}
根据Wheeler的描述,``the matter tells spacetime how to curve, and the spacetime tells matter how to move",可以看到,Einstein场方程是高度耦合在一起的,我们说这样的系统是高度非线性的,因此它的求解是非常困难的一件事情。
然而,我们可以通过在一些特殊情形下对其进行分析,进而得到一些有意义的结论。
一个比较有用的特殊情形,就是{\heiti{弱场近似(weak-field approximation)}},这样可以将场方程线性化, 因此这也被称为{\heiti{线性化引力近似(linearized gravity approximation)}}。
如果加上低速限制条件,我们就可以得到\GR 的Newtonian极限。

\subsection{线性化引力(linerized gravity)}\label{sec:LinGrav}
% Creighton and Anderson ch 2.5.1 
% Schutz p192
让我们考虑如下情形:在原本平直的时空背景上,出现了一个小小的扰动,那么时空度规便偏离了原本的Minkowski 度规$\eta_{\alpha\beta}$
\begin{equation}\label{eq:LinearMetric} 
  g_{\alpha\beta}=  \eta _{\alpha\beta} + h_{\alpha\beta}
\end{equation}
当然,这样的扰动并不大,我们可以在指标升降(见\rprop{prop:RaiseLower})时,近似使用Minkowski 度规$\eta _{\alpha\beta}$而不是真实的度规$g_{\alpha\beta}$。\footnote{注,计算度规的逆变分量时除外,见\cite{Creighton2011}2.127式}

由此可以计算在弱场近似下的Riemann 张量
\begin{equation}\label{eq:RiemannTensorLin}
  R_{\alpha\beta\gamma\delta}= \frac{1}{2}\left(- \frac{\partial^2 h_{\beta\delta}}{\partial x^\alpha \partial x^\gamma} + \frac{\partial^2 h_{\beta\gamma}}{\partial x^\alpha \partial x^\delta} + \frac{\partial^2 h_{\alpha\delta}}{\partial x^\beta \partial x^\gamma} - \frac{\partial^2 h_{\alpha\gamma}}{\partial x^\beta \partial x^\delta}  \right) + \mathcal{O}(h^2)
\end{equation}
其中,$h = h_\mu^{~\mu}$是$h_\mu^{~\nu}$的{\textbf{迹(trace)}}

由此,线性化的Ricci张量可以写为
\begin{equation}\label{eq:RicciTensorLin}
\begin{array}{r@{}l}
  R_{\alpha\beta} &{}= R_{\alpha\mu\beta} ^{\quad\ \mu}\\
                  &{}= \frac{1}{2}\left(- \frac{\partial^2 h}{\partial x^\alpha \partial x^\beta} + \frac{\partial^2 h^\mu_{~\beta}}{\partial x^\alpha \partial x^\mu} + \frac{\partial^2 h_{\alpha}^{~\mu}}{\partial x^\mu \partial x^\beta} - \eta^{\mu\nu}\frac{\partial^2 h_{\alpha\beta}}{\partial x^\mu \partial x^\nu}  \right)+ \mathcal{O}(h^2)
\end{array}
\end{equation}

通过变量代换,使用trace-reversed perturbation $\bar{h}_{\alpha\beta}$
\begin{equation}\label{eq:TraRev} 
  \bar{h}_{\alpha\beta} \equiv h_{\alpha\beta} - \frac{1}{2}\eta_{\alpha\beta}\bar{h}
\end{equation}
在线性化近似下,可以得到形式相对简化的Einstein张量
\begin{equation}\label{eq:EinsteinTensorLin}
  G_{\alpha\beta}  = \frac{1}{2}\left( \frac{\partial^2 \bar{h}^\mu_{~\beta}}{\partial x^\alpha \partial x^\mu} + \frac{\partial^2 \bar{h}^\mu_{~\alpha}}{\partial x^\mu \partial x^\beta} + \eta^{\mu\nu}\frac{\partial^2 \bar{h}_{\alpha\beta}}{\partial x^\mu \partial x^\nu} - \eta_{\alpha\beta}\frac{\partial^2 \bar{h}^{\mu\nu}}{\partial x^\mu \partial x^\nu}  \right)+ \mathcal{O}(h^2)
\end{equation}

由此,可以将现行化的Einstein场方程化为
\begin{equation}\label{eq:EinsteinEqLin}
  - \eta^{\mu\nu} \frac{\partial^2 \bar{h}_{\alpha\beta}}{\partial x^\mu \partial x^\nu} 
  - \eta_{\alpha\beta}\frac{\partial^2 \bar{h}^{\mu\nu}}{\partial x^\mu \partial x^\nu} 
  + \frac{\partial^2 \bar{h}^\mu_{~\beta}}{\partial x^\alpha \partial x^\mu} 
  + \frac{\partial^2 \bar{h}^{\mu}_{~\alpha}}{\partial x^\mu \partial x^\beta}  
  + \mathcal{O}(h^2) = \frac{16\pi G}{c^4} T_{\alpha\beta}
\end{equation}

值得注意的是,等式的第一项可以写为$-\Box\bar{h}_{\alpha\beta}$,其中$\Box$符号为d’Alembertian算符,也就是在平直时空中的波算符。

但是,这个等式依然过于冗长。
但实际上,\GR 是包含一定冗余的自由度的,可以通过选取合理的规范,使得公式\ref{eq:EinsteinEqLin}更简洁。
通常,可以选取所谓Lorenz规范\footnote{注意,很多书本中错误地写作Lorentz规范,其实Lorenz和Lorentz是位科学家家},即
\begin{equation}\label{eq:LorenzGauge}
  \bar{h}^{\mu\nu}_{~~,\nu} = 0
\end{equation}
在这样的形式下,公式\ref{eq:EinsteinEqLin}中,等式左边的后三项均变为零。
我们也就得到了Lorenz规范下的Einstein场方程
\begin{equation}\label{eq:EinsteinEqLorenzGauge}
  -\Box\bar{h}_{\alpha\beta}= \frac{16\pi G}{c^4} T_{\alpha\beta}
\end{equation}

\subsection{Newtonian极限}
% Creighton and Anderson ch 2.5.2 
% Schutz ch 8.4
% Will ch 5.5.7
在低速($v \ll c$)情形下,Newtonian力学可以看做是\SR 的极限情形。
类似地,在弱场($h \ll 1$)、低速($v \ll c$)条件下,Newtonian引力也应该可以看做是\GR 的极限情形。
这一节里,我们将展示这一点。

在Newtonian极限下,我们可以得到如下条件
\begin{equation}\label{eq:TmunuNewton}
\begin{array}{r@{}l}
  T_{00}/c^4      &{} = \rho \qquad ({\rm mass~energy~density}) \\
  | T_{0i}| /c^3  &{} \approx \rho (v/c) \ll T_{00}/c^4 \qquad ({\rm slow~motion}~v \ll c) \\
  | T_{ij}| /c^2  &{} \approx p/c^2 \& \rho(v/c)^2 \ll T_{00}/c^4 \qquad ({\rm small~internal~stresses})
\end{array}
\end{equation}

在低速情形下,$\partial /\partial t \approx v \partial /\partial x $是小量,由此d’Alembertian算符可以近似为空间Laplac算符$\Box \to \nabla^2$。
因此,在Newtonian极限下,场方程退化为
\begin{equation}\label{eq:EinsteinEqNewton}
\begin{array}{r@{}l}
  \nabla^2 \bar{h}_{00} &{}= -16\pi G \rho\\
  \nabla^2 \bar{h}_{0i} &{}= 0 \\
  \nabla^2 \bar{h}_{ij} &{}= 0 \\
\end{array}
\end{equation}
这个方程组可以得到平凡解$\bar{h}_{0i} = 0$和$\bar{h}_{ij}= 0 $。
对应公式\ref{eq:NewtonPoisson},可以发现非平凡解$\bar{h}_{00}=2h_{00}=-4\Phi$。
这样,我们通过Newtonian极限,印证了场方程中系数$8\pi G/c^4$的合理性。

更进一步,可以得到
\begin{equation}\label{eq:NewtonianMetric}
  g_{\mu\nu} ={\begin{pmatrix}
    -1-2\Phi/c^2 & 0 & 0 & 0\\
    0 & 1-2\Phi/c^2 & 0 & 0\\ 
    0 & 0 & 1-2\Phi/c^2 & 0\\
    0 & 0 & 0 & 1-2\Phi/c^2\end{pmatrix}} + \mathcal{O}(\Phi^2/c^4)
\end{equation}


\section{横向无迹规范(Transverse Traceless gauge)}
\subsection{引力波的传播}
% Schutz cp 9.1
% Maggorie1 p6
从公式\ref{eq:EinsteinEqLorenzGauge}中可以看到,在Lorenz规范下,由平直时空背景上的线性微绕引起的运动方程是波动方程,其中能量-动量张量是源项。
在真空中,度规微绕的解就变成了波,这就是引力波。
具体来说,该方程的解具有如下形式:
\begin{equation}\label{eq:GWsolution} 
  \bar{h}^{\alpha\beta} =  A^{\alpha\beta} \exp({\rm i} k_\alpha x^\alpha)
\end{equation}
注意,其中${k_\alpha}$是1-形式的(实)常数分量,而$A^{\alpha\beta}$是某个张量的常数成分。

不难证明,%Schutz eq. 9.1-9.5
$k_\alpha$是一个{\textbf{零(null)}}1-形式,换句话说,矢量$k_\alpha$是零矢量(类光的),它与光子的世界线相切。
更进一步的,可以证明%Schutz eq. 9.5-9.9
引力波的传播速度即光速,传播过程中不包含色散。%Schutz eq. 9.5-9.10

不难发现,这个解是平面波,也就是说,在满足
\begin{equation}\label{eq:PlaneWave} 
  k_\alpha x^\alpha = k_0+\textbf{k}\cdot\textbf{x} =  {\rm const.}
\end{equation}
的超曲面上,$\bar{h}^{\alpha\beta} $是一个常数。
这其中,$\textbf{k}={k^i}$。
方便起见,可以定义$k^0=\omega$,这样,矢量$\vec{k}$的四维分量可以写成$(\omega,\textbf{k})$。

在推导公式\ref{eq:EinsteinEqLorenzGauge}的过程中,用到了规范条件$\bar{h}^{\alpha\beta}_{\quad,\beta} = 0$。
同时注意,通过对波动解公式\ref{eq:GWsolution}求偏导,有$\bar{h}^{\alpha\beta}_{\quad ,\mu} = {\rm i}k_\mu \bar{h}^{\alpha\beta}$


\subsection{Lorenz规范}
% Schutz p191
% Will ch 5.5.3

\subsection{TT规范下的引力波}
% Maggorie1 1.2
% Creighton and Anderson ch 3.1 
% Schutz p205
% Will ch 11.1.5

\subsection{引力波的效果}
% Will ch 11.1.7-8
% Schutz p206-208
% Jaranowski 5.1-5.2
