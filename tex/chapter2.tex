%# -*- coding: utf-8-unix -*-
%%==================================================

\chapter{引力波方程}
\label{chap2}

\section{\GR 的Newtonian极限}
根据Wheeler的描述,``the matter tells spacetime how to curve, and the spacetime tells matter how to move",可以看到,Einstein场方程是高度耦合在一起的,我们说这样的系统是高度非线性的,因此它的求解是非常困难的一件事情。
然而,我们可以通过在一些特殊情形下对其进行分析,进而得到一些有意义的结论。
一个比较有用的特殊情形,就是{\heiti{弱场近似(weak-field approximation)}},这样可以将场方程线性化, 因此这也被称为{\heiti{线性化引力近似(linearized gravity approximation)}}。
如果加上低速限制条件,我们就可以得到\GR 的Newtonian极限。

\subsection{线性化引力(linerized gravity)}
% Creighton and Anderson ch 2.5.1 
% Schutz p192
让我们考虑如下情形:在原本平直的时空背景上,出现了一个小小的扰动,那么时空度规便偏离了原本的Minkowski $\eta_{\alpha\beta}$。
当然,这样的扰动并不大,我们假设可以

\subsection{Newtonian极限}
% Creighton and Anderson ch 2.5.2 
% Schutz ch 8.4
% Will ch 5.5.7

\section{横向无迹规范}

\subsection{波动解}
% Schutz cp 9.1
% Will ch 5.5.3
% Maggorie1 p6

\subsection{Lorenz规范}
% Schutz p191
% Will ch 5.5.3

\subsection{TT规范下的引力波}
% Maggorie1 1.2
% Creighton and Anderson ch 3.1 
% Schutz p205
% Will ch 11.1.5

\subsection{引力波的效果}
% Will ch 11.1.7-8
% Schutz p206-208
% Jaranowski 5.1-5.2
